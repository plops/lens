\documentclass[columns=2]{scrartcl}
\usepackage{units}
\title{Simple Raytracer in Common Lisp}
\author{Martin Kielhorn}
\begin{document}
\maketitle
\section{Introduction}
We want to design a device consisting of multiple light diodes that
are fired in sequence. LEDs can deliver more light when run in a
pulsed mode. So in principle it should be possible to obtain higher
brightness inside the field of view of a microobjective.

Our optical setup consists of several lenses to reimage the light
emitting area into the back focal plane of the objective and one
movable mirror. The light emitted from the diode is assumed to be
lambert distributed. In order to find good focal length ratios and the
diameters of the lenses the optical system is simulated by raytracing
(neglecting aberrations).

The raytracer consists of several distinct modules whose function and
implementation will be described in this document.

\end{document}