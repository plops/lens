\documentclass[twocolumn,DIV20]{scrartcl}
\usepackage{units}
\usepackage{verbatim}
\title{Simple Raytracer in Common Lisp}
\author{Martin Kielhorn}
\date{2010-09-05}
\begin{document}
\maketitle
\section{Introduction}
We design a device consisting of multiple light diodes that are fired
in sequence. LEDs can deliver more light when run in a pulsed mode. So
in principle it should be possible to obtain higher brightness inside
the field of view of a microobjective.

Our optical setup consists of several lenses to reimage the light
emitting area into the back focal plane of the objective and one
movable mirror. The light emitted from the diode is assumed to be
lambert distributed. In order to find good focal length ratios and the
diameters of the lenses the optical system is simulated by raytracing
(neglecting aberrations).

The raytracer consists of several distinct modules whose function and
implementation will be described in this document.

\section{Vector}
The fundamental datastructure are three-dimensional vectors. In the
program they are represented as an array with three elements. The type
of the elements (called vec-float) can be set to either single-float
or double-float. For the simulation the latter choice is
endoresed. 
\subsection{Single or double-float?}
However, some effort has been put into making it possible
to switch to single-float at compile time. This effort consists mainly
of replacing numerical constants like 1d0 with the constant +one+ that
is automatically set to 1s0 or 1d0 depending on the type vec-float.

A three-dimensional vector can be created with make-vec. This expects
zero to three arguments and defaults to zero. If all arguments are
constant numerals a macrocall like \verb!(v 0 2.3)! should be
used. It coerces the arguments to be of the right type for vec-float,
obliviating cumbersome calls like \verb!(make-vec :y 2.3d0)!
(which would only work for double-float) or the even more cumbersome
notation \verb!(make-vec :y #.(* +one+ 2.3d0))!
which would enforce conversion to double-float at compile time if
necessary.

The main motivation to introduce the extra type vec-float is error
estimation. If running a simulation with single-float gives the same
result as with double-float we can be quite sure that round-off errors
are not skewing our result.

\subsection{Operations on vectors}
Addition v+, subtraction v-, multiplication with scalar (v* scalar v),
dot product v. and cross product vx are working as expected. The
euclidean length of a vector is given by norm and the result of the
function normalize is a unit vector with the same direction as the
1input vector.

\subsection{Matrix operations}
The function m takes 9 parameters and creates a new $3\times3$
matrix. The function \verb!(rotation-matrix angle v)! takes an angle
in radians and a unit vector and defines the corresponding rotation
around an axis. The function \verb!(m* m v)! applies a matrix to a
vector.

\subsection{Macros}
Sometimes the macro with-arrays (found in Nikodemus' raylisp) is
handy. It obliviates the need to write aref for function accesses.

\section{Graphical User Interface}
The macro with-gui opens a window calling the body repeatedly to draw
the contents. It makes sense to call a function draw inside
with-gui. Everytime draw is updated from within slime the changes are
immediately visible on the screen.

\section{Optics}
This module contains the basic raytracer.
\subsection{Objects}
The objects plane, mirror, lens, objective and ray are implemented as
classes. A plane is defined by a contained point (called center) and a
normal. A mirror additionally contains a radius and a lens its
lens-radius and focal-length. An objective is a subclass of length
augmented with immersion-index, numerical-aperture and bfp-radius. The
lens-radius of an objective isn't important and should be set to 10
times bfp-radius. A ray consists of a start position and a direction
vector. In general the direction of a ray should have length one.
\subsection{Raytrace}
The generic methods intersect, refract and reflect do the obvious
calculation between a ray and another object resulting in a new
ray. If the ray doesn't pass through an aperture or hits an objective
in a too steep angle the condition ray-lost is signalled. All methods
check that directions are unit vectors. Note that the resulting
direction of raytrace on a lens isn't normalized.
 
\end{document}